%%%%%%%%%%%%%%%%%%%%%%%%%%%%%%%%%%%%%%%%%
% Journal Article
% LaTeX Template
% Version 1.4 (15/5/16)
%
% This template has been downloaded from:
% http://www.LaTeXTemplates.com
%
% Original author:
% Frits Wenneker (http://www.howtotex.com) with extensive modifications by
% Vel (vel@LaTeXTemplates.com)
%
% License:
% CC BY-NC-SA 3.0 (http://creativecommons.org/licenses/by-nc-sa/3.0/)
%
%%%%%%%%%%%%%%%%%%%%%%%%%%%%%%%%%%%%%%%%%

%----------------------------------------------------------------------------------------
%	PACKAGES AND OTHER DOCUMENT CONFIGURATIONS
%----------------------------------------------------------------------------------------

%\documentclass{article}
\documentclass[oneside,twocolumn]{article}

\usepackage{blindtext} % Package to generate dummy text throughout this template 
\usepackage{multicol}
\usepackage[sc]{mathpazo} % Use the Palatino font
\usepackage[T1]{fontenc} % Use 8-bit encoding that has 256 glyphs
\linespread{1.05} % Line spacing - Palatino needs more space between lines
\usepackage{microtype} % Slightly tweak font spacing for aesthetics

\usepackage[english]{babel} % Language hyphenation and typographical rules

\usepackage[hmarginratio=1:1,top=32mm,columnsep=20pt]{geometry} % Document margins
\usepackage[hang, small,labelfont=bf,up,textfont=it,up]{caption} % Custom captions under/above floats in tables or figures
\usepackage{booktabs} % Horizontal rules in tables

\usepackage{lettrine} % The lettrine is the first enlarged letter at the beginning of the text

\usepackage{enumitem} % Customized lists
\setlist[itemize]{noitemsep} % Make itemize lists more compact

\usepackage{abstract} % Allows abstract customization
\renewcommand{\abstractnamefont}{\normalfont\bfseries} % Set the "Abstract" text to bold
\renewcommand{\abstracttextfont}{\normalfont\small\itshape} % Set the abstract itself to small italic text

\usepackage{titlesec} % Allows customization of titles
\renewcommand\thesection{\Roman{section}} % Roman numerals for the sections
\renewcommand\thesubsection{\roman{subsection}} % roman numerals for subsections
\titleformat{\section}[block]{\large\scshape\centering}{\thesection.}{1em}{} % Change the look of the section titles
\titleformat{\subsection}[block]{\large}{\thesubsection.}{1em}{} % Change the look of the section titles

\usepackage{fancyhdr} % Headers and footers
\pagestyle{fancy} % All pages have headers and footers
\fancyhead{} % Blank out the default header
\fancyfoot{} % Blank out the default footer
%\fancyhead[C]{Running title $\bullet$ May 2016 $\bullet$ Vol. XXI, No. 1} % Custom header text
\fancyfoot[RO,LE]{\thepage} % Custom footer text

\usepackage{titling} % Customizing the title section

\usepackage{hyperref} % For hyperlinks in the PDF

\usepackage{listings}
\usepackage[dvipsnames]{xcolor}
\definecolor{codegreen}{rgb}{0,0.6,0}
\definecolor{codegray}{rgb}{0.5,0.5,0.5}
\definecolor{codepurple}{rgb}{0.58,0,0.82}
\definecolor{backcolour}{rgb}{1,1,1}
\lstdefinestyle{mystyle}{
    backgroundcolor=\color{backcolour},   
    commentstyle=\color{codegreen},
    keywordstyle=\color{magenta},
    numberstyle=\tiny\color{codegray},
    stringstyle=\color{codepurple},
    basicstyle=\ttfamily\footnotesize,
    breakatwhitespace=false,         
    breaklines=true,                 
    captionpos=b,                    
    keepspaces=true,                 
    numbers=left,                    
    numbersep=5pt,                  
    showspaces=false,                
    showstringspaces=false,
    showtabs=false,                  
    tabsize=2
}
\renewcommand{\lstlistingname}{Código}% Listing -> Algorithm
\lstset{style=mystyle}

%----------------------------------------------------------------------------------------
%	TITLE SECTION
%----------------------------------------------------------------------------------------

\setlength{\droptitle}{-4\baselineskip} % Move the title up

\pretitle{\begin{center}\Huge\bfseries} % Article title formatting
\posttitle{\end{center}} % Article title closing formatting
\title{Wavefront Propagation} % Article title
\author{%
\textsc{Luis Alberto Ballado Aradias} \\%\thanks{A thank you or further information} \\[1ex] % Your name
\normalsize Cinvestav Unidad Tamaulipas \\ % Your institution
\normalsize \href{mailto:luis.ballado@cinvestav.mx}{luis.ballado@cinvestav.mx} % Your email address
%\and % Uncomment if 2 authors are required, duplicate these 4 lines if more
%\textsc{Jane Smith}\thanks{Corresponding author} \\[1ex] % Second author's name
%\normalsize University of Utah \\ % Second author's institution
%\normalsize \href{mailto:jane@smith.com}{jane@smith.com} % Second author's email address
}
\date{\today} % Leave empty to omit a date
\renewcommand{\maketitlehookd}{%
\begin{abstract}
\noindent One of the major applications of mobile robots is to create models of the environment they traverse using sensor data; this process is known as mapping. Military applications of this technology are obvious. Visualize a robot that somehow enters a vacant building in hostile territory. For example, it could be thrown through an open window, crawl through drain pipes, or climb up the side of the building. Once inside, the robot can traverse the hallways and create a map showing doors, hall crossing, stairways, and other features.

The most commonly used approaches to mapping are termed "grid based or metric mapping and topological mapping".
Metric or quantitative maps, as the name implies, are based on measurements of the space they map. An indoor metric map may include the lengths of wall sections, door-opening widths, hallways widths, distances to intersections, and so forth. A typical metric navigation instruction might be "Move 45 meters in north direction, then turn 30º clockwise and move another 65 meters.
Path planning in metrically mapped spaces usually includes the designation of a number of way points at specific (x,y) locations, connected by straight-line segments. Paths can then be selected on the basis of some optimization criterion.
A widely used method of generationg a metric map is to cover the environment to be mapped with an evently spaced grid. Each cell in the grid is then filled with one or more values that represent the presence or absence of an obstacle (which could be another robot or a human).Grid-based mapping was first proposed in the 1980's  by Elfes

\end{abstract}
}

%----------------------------------------------------------------------------------------

\begin{document}

% Print the title
\maketitle

%----------------------------------------------------------------------------------------
%	ARTICLE CONTENTS
%----------------------------------------------------------------------------------------


    
\section{Introduction}

\lettrine[nindent=0em,lines=3]{L} orem ipsum dolor sit amet, consectetur adipiscing elit.
\blindtext % Dummy text

\blindtext % Dummy text

%------------------------------------------------

\newpage
\onecolumn
\section{Serial Algorithm}

All human things are subject to decay. And when fate summons, Monarchs must obey.
\begin{lstlisting}[language=Python, caption=Configuración Dash]
  import dash_bootstrap_components as dbc # importamos el paquete
  
  external_stylesheets = [dbc.themes.BOOTSTRAP, dbc.icons.BOOTSTRAP] # configuramos que tome los estilos de bootstrap asi como los iconos
  
  app = Dash(
  __name__,
  meta_tags=[{"name": "viewport", "content": "width=device-width, initial-scale=1"}],
  external_stylesheets=external_stylesheets,
  title="Distribuciones"
  )
  
  app._favicon = ("assets/favicon.ico") # colocamos un favicon
  
  d = DistribucionFactory() # Se inicializa el objeto abstracto de las distribuciones
  
\end{lstlisting}


\blindtext\blindtext

%------------------------------------------------

\twocolumn
\section{Results}

\begin{table}
  \caption{Example table}
  \centering
  \begin{tabular}{llr}
    \toprule
    \multicolumn{2}{c}{Name} \\
    \cmidrule(r){1-2}
    First name & Last Name & Grade \\
    \midrule
    John & Doe & $7.5$ \\
    Richard & Miles & $2$ \\
    \bottomrule
  \end{tabular}
\end{table}

\blindtext % Dummy text

\begin{equation}
  \label{eq:emc}
  e = mc^2
\end{equation}

\blindtext % Dummy text

%------------------------------------------------

\section{Conclusions}

\subsection{Subsection One}

A statement requiring citation \cite{Figueredo:2009dg}.
\blindtext % Dummy text

\subsection{Subsection Two}

\blindtext % Dummy text

%----------------------------------------------------------------------------------------
%	REFERENCE LIST
%----------------------------------------------------------------------------------------

\begin{thebibliography}{99} % Bibliography - this is intentionally simple in this template
  
\bibitem[George A. Bekey, 2005]{GeorgeABekey:2005dg}
  George A. Bekey - Autonomus Robots From Biological Inspiration to Implementation and Control - MIT Press (2005).
  \newblock {\em ISBN}, 0-262-02578-7

\bibitem[Ronald C. Arkin, 1998]{RonaldCArkin:1998dg}
  Ronald C. Arkin - Behavior Based Robotics - MIT Press (1998).
  \newblock {\em ISBN}, 978-0-262-01165-5

\bibitem[Ronald C. Arkin, 1998]{RonaldCArkin:1998dg}
  Ronald C. Arkin - Behavior Based Robotics - MIT Press (1998).
  \newblock {\em ISBN}, 978-0-262-01165-5


\bibitem[Ronald C. Arkin, 1998]{RonaldCArkin:1998dg}
  Ronald C. Arkin - Behavior Based Robotics - MIT Press (1998).
  \newblock {\em ISBN}, 978-0-262-01165-5

\bibitem[Ronald C. Arkin, 1998]{RonaldCArkin:1998dg}
  Ronald C. Arkin - Behavior Based Robotics - MIT Press (1998).
  \newblock {\em ISBN}, 978-0-262-01165-5
  
\end{thebibliography}

%----------------------------------------------------------------------------------------

\end{document}
